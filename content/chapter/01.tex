%!TEX root = ../../main.tex

\chapter{Einleitung}
Das Reich der Tier besteht aus sehr vielen verschiedenen Spezien, welche alle unterschiedliche Eigenschaften besitzen.
Manche Tiere können fliegen, während andere schwimmen. Einige Tiere haben Fell, andere wiederrum Federn oder Schuppen.
Diese Vielzahl an Eingenschaften ist gut dafür geeignet, um die Grundzüge der \acf{OOP} aufzuzeigen.

Zu diesem Zweck soll ein Programm entwickelt werden, welches die Funktionsweiße der \acs{OOP} darstellt.
Dafür wird in der Programmiersprache Python ein Mini-Zoo aufgebaut, welcher unterschiedliche Tierarten umfasst.
Hier werden die speziellen Eigenschaften der Tiere durch Klassen und Vererbungen modelliert. Besonders für die Modellierung,
ist die Verwendung von Überschreibungen von Funktionen. Zusätzlich zu den \acs{OOP}-spezifischen Code-Bausteinen,
wie Klassen und Vererbungen, wird in dem Mini-Zoo noch Fehlerbehandlung durch Ausnahmen und Protokollierung eingebaut.
Des Weiteren erfolgen Unit-Tests um die korrekte Funktionsweiße festzustellen, sowie Code-Dokumentation.

Im Hauptteil der Hausarbeit wird zunächst das Grundgerüst des Software-Projektes näher erläutert. Hier wird
sowohl die Vorarbeit, als auch der strukurelle Aufbau näher erläutert. Anschließend erfolgt eine theoretische 
Betrachtung der \acs{OOP}-Konzepte, welche darauf auf das Mini-Zoo-Programm angewendet werden. Darauf erfolgt 
die Fehlerbehandlung und die Unit-Tests. 

Das Programm wird mit Hilfe von Code-Ausschnitten und Abbildungen näher erläutert. Diese werden gezielt gewählt,
um die erläuterten Aussagen und Vorgehensweisen darzustellen.


%!TEX root = ../../main.tex

\chapter{Grundgerüst}
Im folgenden Abschnitten wird darauf eingegangen, welche Entwicklungsumgebung verwendet und eine virtuelle Umgebung in aufgesetzt wird. Darauf
wird die Versionsverwaltung mit Git erklärt und wie diese in PyCharm integriert ist. Der Abschluss dieses Kapitel
betrachtet die Ordnerstruktur des Code-Projektes.

\section{Einrichtung der Entwicklungsumgebung}
Als \acf{IDE} wird PyCharm von JetBrains verwendet. PyCharm bietet viele Möglichkeiten, um die Entwicklung
von Python-Programmen zu vereinfachen. Dazu gehören Debugging-Werkzeuge, Code-Vervollständigung, die Integration von Versionskontrolle
und die Option des Refactoring (\cite{pyCharmGettingStarted}).

\section{Einrichtung einer virtuellen Umgebung}
\glqq Eine virtuelle Umgebung oder Virtualenv Python Version ermöglicht es, mehrere parallele Instanzen
 des Python-Interpreters zu starten. Jede dieser Instanzen hat ihren 
 eigenen Satz an Paketen und ihre eigenen Konfigurationen. 
 Jede virtuelle Umgebung enthält auch eine Kopie des Python-Interpreters.\grqq (\cite{team_virtualenv_2023}).

\section{Git-Integration}

\section{Ordner-Struktur}